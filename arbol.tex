\documentclass[10pt,letterpaper]{article}
\usepackage[utf8]{inputenc}
\usepackage[spanish]{babel}
\usepackage{tikz-qtree}
\usepackage{multicol}
\usepackage{multirow}
\usepackage{xspace}
\usepackage{color, colortbl}
\usepackage{underscore}
\usepackage{tabu}
\usepackage{url}
\usepackage{ragged2e}
\usepackage{verbatim}
\usepackage{mathdots} 
\usepackage{amsmath, amssymb, amsbsy, amsfonts} 
\usepackage[left=3.5cm,right=3.5cm,top=3.5cm,bottom=3.5cm]{geometry}
\setlength{\parskip}{\baselineskip}
\begin{document} 
    \begin{titlepage} 
    \newcommand{\HRule}{\rule{\linewidth}{0.5mm}} 
    \center   
    \textsc{\Huge Instituto Tecnológico de Costa Rica}\\[1.5cm] 
    \textsc{\normalsize PROYECTO DE INVESTIGACIÓN DE OPERACIONES}\\[0.5cm] 
    \textsc{\normalsize PROYECTO 2}\\[0.5cm] 
    \HRule\\[0.4cm] 
    {\huge\bfseries \vspace{1cm} Árboles Binarios}\\[0.4cm] 
    \HRule\\[2cm] 
    \textbf{\Large Estudiantes}\\[0.5cm] 
        \begin{minipage}{0.4\textwidth} 
        \begin{flushleft} 
            \large 
            Jason Barrantes Arce 
            \textsc{2015048456} 
        \end{flushleft} 
    \end{minipage} 
    ~ 
    \begin{minipage}{0.4\textwidth} 
        \begin{flushright} 
           	\large 
            Steven Bonilla Zúñiga 
            \textsc{2015056296} 
        \end{flushright} 
    \end{minipage} 
   \newline \newline 
   \newline  
   \textbf{\Large Profesor}\\[0.5cm] 
    \textsc{\normalsize Francisco Torres Rojas}\\[0.5cm] 
    \end{titlepage} 
    
\titlepage{\textbf{Modo Ejemplo:}} \newline \newline 
        Se resolverá un problema general por medio de diversos algoritmos que nos 
        permitan encontrarle una solución. 
        El problema es sobre árboles de búsqueda binaria. \\ 
        Hay que ordenar una serie de valores o llaves en forma de árbol binario, de manera que 
        el nivel de búsqueda promedio sea el óptimo. 
        Tenemos una serie de llaves que quieren acomodarse en una estructura de árbol, 
        pero esas llaves deben tener un respectivo peso, que determinará su probabilidad 
        y un cáracter único que será asignado de manera ascendente. \ \ \newline \newline 
        Restricciones: 
        \begin{itemize} 
        \item \textbf{Estructura:} Se formará un árbol binario óptimo. 
        \item \textbf{Llaves:} Se generarán 6 llaves de forma ascendente. 
        \item \textbf{Peso:} Varía entre $1 \leq C{i} \leq 1000$. 
        \item Los caracteres ASCII varían. 
        \end{itemize} 
        Los dos algoritmos que vamos a implementar son: 
        \begin{itemize} 
        \item \textbf{Algoritmo de Búsqueda Dinámica:} Algoritmo para el ABB óptimo. 
        \item \textbf{Algoritmo Greedy Básico:} Cada vez se escoge la llave de máxima probabilidad 
         para que sea la raíz del árbol. 
        \end{itemize} 
        \ En el caso de programación dinámica ya que nuestro objetivo es minimizar el costo promedio de la búsqueda, usamos la fórmula: 
        \[ \textsc{\normalsize MIN(Z)}\\[0.5cm] = \sum_{i=1}^{n}c_{i}p_{i} \] 
        \ Que está sujeto a:  
        \[ \sum p_{i} \equiv 1 \] 
        \ Con cada $c_{i}$ = 1, 2, ... n  
        \ \ \newline \newline 
        
Se muestra a continuación la tabla de objetos con su respectivo costo (peso) y probabilidad 
        que fueron asignados aleatoriamente cumpliendo con las restricciones: 
\definecolor{Gray}{gray}{0.9}
\definecolor{LightCyan}{rgb}{0.88,1,1}
\begin{center}
\begin{tabu} to 0.6\textwidth { | X[l] | X[l] | X[l] | } 
\hline
\rowcolor{Gray}
\textbf{Letra} & \textbf{Peso} & \textbf{Probabilidad}\\
\hline
Object A & 891 & 0.28 \\
\hline
Object B & 562 & 0.18 \\
\hline
Object C & 536 & 0.17 \\
\hline
Object D & 67 & 0.02 \\
\hline
Object E & 792 & 0.25 \\
\hline
Object F & 360 & 0.11 \\
\hline
\end{tabu} \\
\end{center}
\section{Algoritmo AAB Óptimo} 
        Es un problema de solución de búsqueda óptima:  
        En nuestro caso, queremos minimizar el costo de la búsqueda promedio. 
        Para solucionar el problema vamos a hacer uso de una tabla (n+1) x (n+1), donde n es la cantidad 
        de objetos o llaves disponibles. 
        Como se menciona en las restricciones del problema $n = 6$ por lo que tendremos 
        dos tablas (7x7). La tabla A donde estará el costo promedio y la tabla R donde estará 
        los índices de búsqueda más rápidos. \newline \newline \newline 
        \textbf{\Large Fórmula Matemática} 
        \[ \textsc{\normalsize MIN(Z)}\\[0.5cm] = \sum_{i=1}^{6}c_{i}p_{i} \] 
        \ Sujeto a:  
        \[ \sum p_{i} \leq 1 \] 
        \ En otras palabras tendremos 
\[ (p_{1} \approx 0.2777)+(p_{2} \approx 0.1752)+(p_{3} \approx 0.1671)+(p_{4} \approx 0.0209)+(p_{5} \approx 0.2469)+(p_{6} \approx 0.1122) = 1 \]
\newline Ahora proseguimos realizando la tabla dinámica.
\newline \newline \newline \textbf{Tabla A: }
\definecolor{Gray}{gray}{0.5}
\definecolor{GreenBlack}{RGB}{2,80,0}
\begin{center}
\begin{tabu} to 1.0\textwidth { | c | c | c | c | c | c | c | c | }
\hline
\cellcolor{Gray}\color{black}\textbf{X} & \cellcolor{Gray}\color{black}\textbf{0} & \cellcolor{Gray}\color{black}\textbf{1} & \cellcolor{Gray}\color{black}\textbf{2} & \cellcolor{Gray}\color{black}\textbf{3} & \cellcolor{Gray}\color{black}\textbf{4} & \cellcolor{Gray}\color{black}\textbf{5} & \cellcolor{Gray}\color{black}\textbf{6} \\ 
\hline
\cellcolor{Gray}\color{black}1 & 0.00 & 0.28 & 0.63 & 1.06 & 1.13 & 1.80 & 2.14 \\ 
\hline
\cellcolor{Gray}\color{black}2 &  --  & 0.00 & 0.18 & 0.51 & 0.56 & 1.07 & 1.39 \\ 
\hline
\cellcolor{Gray}\color{black}3 &  --  &  --  & 0.00 & 0.17 & 0.21 & 0.64 & 0.87 \\ 
\hline
\cellcolor{Gray}\color{black}4 &  --  &  --  &  --  & 0.00 & 0.02 & 0.29 & 0.51 \\ 
\hline
\cellcolor{Gray}\color{black}5 &  --  &  --  &  --  &  --  & 0.00 & 0.25 & 0.47 \\ 
\hline
\cellcolor{Gray}\color{black}6 &  --  &  --  &  --  &  --  &  --  & 0.00 & 0.11 \\ 
\hline
\cellcolor{Gray}\color{black}7 &  --  &  --  &  --  &  --  &  --  &  --  & 0.00 \\ 
\hline
\end{tabu} \\
\end{center}
\textbf{Tabla R: }
\begin{center}
\begin{tabu} to 1.0\textwidth { | l | l | l | l | l | l | l | l | }
\hline
\cellcolor{Gray}\color{black}\textbf{X} & \cellcolor{Gray}\color{black}\textbf{0} & \cellcolor{Gray}\color{black}\textbf{1} & \cellcolor{Gray}\color{black}\textbf{2} & \cellcolor{Gray}\color{black}\textbf{3} & \cellcolor{Gray}\color{black}\textbf{4} & \cellcolor{Gray}\color{black}\textbf{5} & \cellcolor{Gray}\color{black}\textbf{6} \\ 
\hline
\cellcolor{Gray}\color{black}1 &  0  &  1  &  1  &  2  &  2  &  3  &  3  \\ 
\hline
\cellcolor{Gray}\color{black}2 &  -  &  0  &  2  &  2  &  3  &  3  &  5  \\ 
\hline
\cellcolor{Gray}\color{black}3 &  -  &  -  &  0  &  3  &  3  &  5  &  5  \\ 
\hline
\cellcolor{Gray}\color{black}4 &  -  &  -  &  -  &  0  &  4  &  5  &  5  \\ 
\hline
\cellcolor{Gray}\color{black}5 &  -  &  -  &  -  &  -  &  0  &  5  &  5  \\ 
\hline
\cellcolor{Gray}\color{black}6 &  -  &  -  &  -  &  -  &  -  &  0  &  6  \\ 
\hline
\cellcolor{Gray}\color{black}7 &  -  &  -  &  -  &  -  &  -  &  -  &  0  \\ 
\hline
\end{tabu} \\
\end{center}
El algoritmo tarda aproximadamente: 0.003000 segundos en ejecutarse.

El árbol de búsqueda binario queda representado de la siguiente forma: 
\newline \newline 
\tikzset{every tree node/.style={minimum width=5em,draw,circle},
     blank/.style={draw=none},
     edge from parent/.style=
     {draw,edge from parent path={(\tikzparentnode) -- (\tikzchildnode)}},
     level distance=3cm}
\begin{tikzpicture}
\Tree
[.\node[]{C}; 
\edge[]; 
[.\node[]{A}; 
\edge[blank]; \node[blank]{};
\edge[]; 
[.\node[]{B}; 
\edge[blank]; \node[blank]{};
\edge[blank]; \node[blank]{};
]
]
\edge[]; 
[.\node[]{E}; 
\edge[]; 
[.\node[]{D}; 
\edge[blank]; \node[blank]{};
\edge[blank]; \node[blank]{};
]
\edge[]; 
[.\node[]{F}; 
\edge[blank]; \node[blank]{};
\edge[blank]; \node[blank]{};
]
]
]
\end{tikzpicture}
\newpage
\section{Algoritmo Greedy Básico} 
        Es un algoritmo que soluciona problemas que a primera vista parece ser 
        óptimo. Es característico porque es muy sencillo de entender y explicar. 
        Se escoge la llave de máxima probabilidad para que sea la raíz del árbol, las 
        demas se separan en dos grupos: las menores que la raíz y las mayores. Este 
        procedimiento se repite recursivamente en cada grupo. 
        \[ Llave_{i} = (Peso, Probabilidad), i = 0...n \]
\[ K_{A} = (891, 0.28), K_{B} = (562, 0.18), K_{C} = (536, 0.17), K_{D} = (67, 0.02), K_{E} = (792, 0.25), K_{F} = (360, 0.11) \]
\newline Ahora proseguimos realizando la tabla greedy básico.
\textbf{Tabla R: }
\begin{center}
\begin{tabu} to 1.0\textwidth { | l | l | l | l | l | l | l | l | }
\hline
\cellcolor{Gray}\color{black}\textbf{X} & \cellcolor{Gray}\color{black}\textbf{0} & \cellcolor{Gray}\color{black}\textbf{1} & \cellcolor{Gray}\color{black}\textbf{2} & \cellcolor{Gray}\color{black}\textbf{3} & \cellcolor{Gray}\color{black}\textbf{4} & \cellcolor{Gray}\color{black}\textbf{5} & \cellcolor{Gray}\color{black}\textbf{6} \\ 
\hline
\cellcolor{Gray}\color{black}1 &  0  &  1  &  -  &  -  &  -  &  -  &  1  \\ 
\hline
\cellcolor{Gray}\color{black}2 &  -  &  0  &  2  &  -  &  2  &  -  &  5  \\ 
\hline
\cellcolor{Gray}\color{black}3 &  -  &  -  &  0  &  3  &  3  &  -  &  -  \\ 
\hline
\cellcolor{Gray}\color{black}4 &  -  &  -  &  -  &  0  &  4  &  -  &  -  \\ 
\hline
\cellcolor{Gray}\color{black}5 &  -  &  -  &  -  &  -  &  0  &  5  &  -  \\ 
\hline
\cellcolor{Gray}\color{black}6 &  -  &  -  &  -  &  -  &  -  &  0  &  6  \\ 
\hline
\cellcolor{Gray}\color{black}7 &  -  &  -  &  -  &  -  &  -  &  -  &  0  \\ 
\hline
\end{tabu} \\
\end{center}
El algoritmo tarda aproximadamente: 0.047000 segundos en ejecutarse.

El árbol de búsqueda binario queda representado de la siguiente forma: 
\newline \newline 
\tikzset{every tree node/.style={minimum width=5em,draw,circle},
     blank/.style={draw=none},
     edge from parent/.style=
     {draw,edge from parent path={(\tikzparentnode) -- (\tikzchildnode)}},
     level distance=3cm}
\begin{tikzpicture}
\Tree
[.\node[]{A}; 
\edge[blank]; \node[blank]{};
\edge[]; 
[.\node[]{E}; 
\edge[]; 
[.\node[]{B}; 
\edge[blank]; \node[blank]{};
\edge[]; 
[.\node[]{C}; 
\edge[blank]; \node[blank]{};
\edge[]; 
[.\node[]{D}; 
\edge[blank]; \node[blank]{};
\edge[blank]; \node[blank]{};
]
]
]
\edge[]; 
[.\node[]{F}; 
\edge[blank]; \node[blank]{};
\edge[blank]; \node[blank]{};
]
]
]
\end{tikzpicture}
\newpage
\end{document}
